\subsection{Diskussion der Teilaspekte}
Unsere Teilaspekte umfassen vier Punkte:
\begin{enumerate}
\item
\textbf{Der Benutzer muss intuitiv mit einer anschaulichen Benutzeroberfläche
zurechtkommen.}
\\
\\
Dieser Punkt wurde recht gut erfüllt, da die Struktur sehr logisch aufgebaut ist und auch anderen ähnlichen Applikationen wie z.B. Quizlet, das für das Lernen von Vokabeln dient, gleicht. Auch haben mehrere Testpersonen die App als einfach und intuitiv zu handhaben bezeichnet. Auch grafisch und im Aufbau entspricht die App den heutigen Standards.
\item
\textbf{Der Benutzer kann sich selber aussuchen, welche Karten er lernen will und
welche nicht.}
\\
\\
Auch dieser Punkt wurde zweifelsohne erfüllt. In einem Menü kann der Benutzer das Set, das er lernen will, zuerst in einer Suchleiste eingeben und dann über einen einfachen Knopfdruck lernen.
\item
\textbf{Dem Benutzer stehen eine Android-OS App und eine Webseite zum Lernen
bereit.}
\\
\\
Dieser Punkt wurde leider nur teilweise erfüllt: Zwar steht dem Benutzer die versprochene Android-App zur Verfügung aber leider kann er auf der Webseite die Karten nicht lernen. Dieser Fakt ist aus dem Wechsel des Frameworks herzuleiten. Wenn wir das Projekt mit dem anfangs gewählten Phonegap abgeschlossen hätten, hätten wir automatisch zur Applikation auch eine funktionierende Webseite erhalten, da Phonegap beides erzeugen kann. Wegen dem Wechsel auf Android Studio wurde uns dieser Punkt leider verwehrt. Wegen mangelnder Zeit konnten wir nicht auch noch eine Webseite entwickeln. Das heisst aber auf keinen Fall, dass die Webseite nicht existiert. Diese hat nun die einfache Funktion einer Homepage, auf der man die Applikation herunterladen kann. Sie ist auf \hyperlink{www.magic2brain.com}{www.magic2brain.com} im Internet auffindbar.
\item
\textbf{Das Backend, bzw. das Programm wird weitgehend autonom (wartungs-
frei) arbeiten. Dabei ist die Problematik zu beachten, dass ständig neue
Spielkarten veröffentlicht werden, welche ins System eingetragen werden
müssen.}
\\
\\
Auch dieser Punkt musste aufgrund mangelnder Zeit vernachlässigt werden, da die Priorität auf der Funktionsfähigkeit der Applikation lag. Das Problem mit den ständig neu erscheinenden Spielkarten wird jetzt ganz einfach durch regelmässige Updates der Applikation gelöst, damit der Benutzer trotzdem immer die neusten Karten lernen kann. Wenn die App einen guten zulauf geniessen wird, ist es sehr wahrscheinlich, dass wir auch nach der Projektabgabe diesen Punkt noch beheben werden.
\end{enumerate}
\subsection{Diskussion des Endergebnisses}
Nur Dank unserer bereits reichlich gesammelten Erfahrung war es möglich, eine solche Applikation umzusetzen, da wir uns nicht mehr stundenlang mit den Grundlagen auseinandersetzen mussten, sondern direkt zu entwickeln beginnen konnten. Trotzdem haben wir uns auf Neuland gewagt und unseren Horizont im Bereich der Entwicklung einer Android-Applikation erweitern können. Wir haben es tatsächlich geschafft, eine ansehnliche Applikation zu erstellen, welche zumindest im grafischen Bereich und in der Benutzerfreundlichkeit dem heutigen Standard entspricht. Bei den Optionen, über die der Benutzer verfügt, mussten wir aber sparen. Unser Entwicklungsteam bestand nur aus 3 Personen und eine Applikation mit reichhaltiger Funktionalität war deshalb in diesem Zeitraum schlichtweg nicht möglich. Die Zeit, die uns für die Projektarbeit zur Verfügung stand war zudem ziemlich knapp bemessen, da wir zum einen auch den Schulalltag bewältigen mussten und zum anderen auch diese Arbeit schreiben mussten. Wenn man aber bedenkt, dass die meisten erfolgreichen Applikationen von einem professionellen Team entwickelt wurden, ist unser Produkt einer Projektarbeit definitiv würdig. Deshalb sehen wir unser Projekt als gelungen an.
\subsection{Diskussion des methodischen Vorgehens}
Was man jedoch als grossen Kritik- und Verbesserungspunkt ansehen muss, ist die Erörterung eines geeigneten Frameworks. Während mehreren Wochen haben wir an einem Projekt mit Phonegap gearbeitet, in das wir viel Zeit und Arbeit investiert haben, nur um dann zu merken, dass Phonegap für unsere Art von Projekt überhaupt nicht geeignet ist. In diesem Fall haben wir im Voraus viel zu wenig Nachforschung betrieben, um das geeignete Framework zu finden. Von dieser Erfahrung haben wir definitiv profitiert. Die Arbeit an sich verlief sonst konfliktfrei mitunter dank der Organisationssoftware GitHub und der guten Aufteilung der Arbeit. GitHub hat uns hierbei nicht nur bei der Applikation sondern auch bei der schriftlichen Arbeit sehr geholfen. Dank der Include-Funktion von LaTeX war es auch möglich, die Projektarbeit in einzelne Teile aufzuteilen wodurch ein paralleles Arbeiten an den Teilen problemlos vonstatten ging und trotzdem jeder von uns immer die aktuellste Version besessen hat. Des Weiteren hat uns auch unsere Erfahrung von Projektarbeiten vom Lehrgang Infcom.ch geholfen, dieses richtig zu organisieren. Auch während des Infcom.ch-Unterrichts mussten wir mehrmals eine technische Arbeit organisieren und konnten dadurch auf die bereits gesammelte Erfahrung zurückgreifen.

