\subsection{Diskussion der Teilaspekte}
Unsere Teilaspekte umfassten vier Punkte: 
\begin{enumerate}
\item
\textbf{Der Benutzer muss Intuitiv mit einer anschaulichen Benutzeroberfläche
zurechtkommen.}
\\
\\
Dieser Punkt wurde recht gut erfüllt, da die Struktur sehr logisch aufgebaut ist und auch anderen ähnlichen Applikationen wie z.B. Quizlet, das für das Lernen von Vokabeln dient, gleicht. Auch haben mehrere Testpersonen die App als einfach und intuitiv zu handhaben bezeichnet. Auch grafisch und im Aufbau entspricht die App dem heutigen Standart.
\item
\textbf{Der Benutzer kann sich selber aussuchen, welche Karten er lernen will und
welche nicht.}
\\
\\
Auch dieser Punkt wurde zweifelsohne erfüllt. In einem Menu kann der Benutzer das Set, das er lernen will, zuerst in einer Suchleiste eingeben und dann über einen einfachen Knopfdruck lernen. 
\item
\textbf{Dem Benutzer stehen eine Android-OS App und eine Webseite zum Lernen
bereit.}
\\
\\
Dieser Punkt wurde leider nur teilweise erfüllt: Zwar steht dem Benutzer die versprochene Android-App zur Verfügung aber leider kann er auf der Webseite die Karten nicht lernen. Dieser Fakt ist aus dem Wechesel des Frameworks herzuleiten. Wenn wir das Projekt mit dem anfangs gewählten Phonegap abgeschlossen hätten, hätten wir automatisch zur Applikation auch eine funktionierende Webseite erhalten, da Phonegap beides erzeugen kann. Wegen dem Wechsel auf Android Studio wurde uns dieser Punkt leider verwehrt. Wegen mangelnder Zeit konnten wir nicht auch noch eine Webseite entwickeln. Das heisst aber auf keinen Fall, dass die Webseite nicht existiert. Die Webseite hat nun die einfache Funktion einer Homepage, auf der man die Applikation herunterladen kann. Sie ist auf \textbf{www.magic2brain.com} im Internet auffindbar.
\item
\textbf{Das Backend, bzw. das Programm wird weitgehend autonom (wartungs-
frei) arbeiten. Dabei ist die Problematik zu beachten, dass ständig neue
Spielkarten veröffentlicht werden, welche ins System eingetragen werden
müssen.}
\\
\\
Auch dieser Punkt musste aufgrund mangelnder Zeit vernachlässigt werden, da die Priorität auf der Funktionsfähigkeit der Applikation lag. Das Problem mit den ständig neu erscheinenden Spielkarten wird jetzt ganz einfach durch regelmässige Updates der Applikation gelöst, damit der Benutzer trotzdem immer die neusten Karten lernen kann. Es ist auch gut möglich, dass wir nach der Projektabgabe diesen Punkt noch beheben werden.
\end{enumerate}
\subsection{Diskussion des Endergebisses}
Nur Dank unserer bereits reichlich gesammelten Erfahrung war es möglich, eine solche Applikation umzusetzen, da wir uns genau nicht mehr stundenlang mit den Grundlagen auseinandersetzen mussten. Trotzdem haben wir uns auf Neuland gewagt und unseren Horizont erweitern können im Bereich der Entwicklung einer Android-Applikation. Wir haben es tatsächlich geschafft, eine ansehliche Applikation zu erstellen, welche zumindest im grafischen Bereich und in der Benutzerfreundlichkeit dem heutigen Standart entspricht. In der Optionen, über die der Beutzer verfügt, mussten wir aber sparen. Unser Entwicklungsteam bestand nur aus 3 Personen und eine Applikation mit reichhaltiger Funktionalität war deshalb schlicht und einfach nicht möglich. Dazu war die Zeit, die uns für die Projektarbeit zur Verfügung stand zu knapp bemessen, da wir zum einen auch den Schulalltag bewältigen mussten und zum anderen auch diese Arbeit hier schreiben mussten. Wenn man aber bedenkt, dass die meisten erfolgreichen Applikationen von einem professionellen Vollzeit Entwicklerteam geschrieben wurden, ist unser Produkt einer Projektarbeit definitv würdig. Deshalb sehen wir unser Projekt als gelungen an. 
\subsection{Diskussion des methodischen Vorgehens}
Was man jedoch als grosser Kritikpunkt einwerfen muss und was man für ein nächstes Projekt besser machen muss ist die Errörterung eines geeigneten Frameworks. Während mehreren Wochen haben wir an einem Projekt mit Phonegap gearbeitet und viel Zeit und Arbeit darin investiert um dann zu merken, dass Phonegap für unsere Art von Projekt überhaupt nicht geeignet ist. In diesem Fall haben wir im Voraus viel zu wenig Nachforschung betrieben, um das geignete Framework zu finden, dies soll uns ein Lehre sein, es beim nächsten Mal besser zu machen. Die Arbeit an sich verlief sonst aber reibunglos mitunter dank der Organisationssoftware GitHub und der guten Aufteilung der Arbeit. GitHub hat uns hierbei nicht nur bei der Appikation sondern aucnh bei der schriftlichen Arbeit sehr geholfen. Dank der Include-Fuktion von LaTeX war es auch möglich, die Projektarbeit in einzelne Teile aufzuteilen wodurch ein paralleles Arbeiten an den Teilen problemlos funktionierte und trotzdem jeder von uns immer die aktuelste Version besass. Des Weiteren hat uns auch unsere Erfahrung von Projektarbeiten vom Lehrgang Infcom geholfen, dieses richtig zu organisieren. Auch während des Infcom-Unterrichts mussten wir mehrmals eine technische Arbeit organisieren und konnten dadurch aus der bereits gesammelten Erfahrung schöpfen.