Es gibt mehrere Möglichkeiten eine Android App zu programmieren. Man kann ein Framework (Eine Art Bibliothek, die als Gerüst funktioniert)  (z.B. "`LibGDX"') nutzen um mit einer vertrauten Programmiersprache und IDE eine App zu entwickeln. Google selbst bietet eine eigene IDE namens "`Android Studio"'. Wir entschieden uns aber anfangs mit dem Framework Phonegap zu arbeiten.


\subsection{Die Arbeit mit Phonegap}
Einer der Vorteile von Phonegap besteht darin, dass es für das Design HTML nutzt. Wir kennen HTML bereits und können damit schnell relativ ansehnliche Benutzeroberflächen erstellen. Als Programmiersprache nutzt Phonegap Javascript. Auch diese kennen wir, weil wir ein Semester lang damit gearbeitet haben. Wir konnten also direkt loslegen. Phonegap bietet zudem die Möglichkeit die App direkt auf dem Smartphone zu betrachten. Diese Funktion ist sehr praktisch, wenn man etwas betrachten will.

\subsection{Der Aufbau der App}
Wir wollten die Arbeit an der App gleichmässig aufteilen. Wenn mehrere Personen an einem Projekt arbeiten, muss man dafür sorgen, dass keine Konflikte entstehen. Solche können entstehen, wenn z.B. mehrere Personen die gleiche Datei bearbeiten. Um dies zu verhindern haben wir jede Activity der App in eine eigene Datei ausgelagert. Es sollte ein Menü geben, von wo aus man auf jede Seite zugreifen konnte. Dieses Menü musste anfangs von einer Person erstellt werden. Daraufhin konnten wir die einzelnen Seiten aufteilen und gleichzeitig arbeiten, ohne dass Konflikte entstehen. Jede Activity sollte eine eigene Funktion haben. z.B. gab es eine Activity für die Abfragen, eine mit Favoriten etc. Mit einem Zurück-Knopf kommt man von jeder Activity zum Menü zurück.

\subsection{Der Wechsel zu "`Android Studio"'}
Wir haben viel Zeit investiert um mit Phonegap eine schöne Benutzeroberfläche zu erstellen. Als wir dann dazu kamen die Funktionen zu programmieren, fiel uns auf was der Nachteil von Phonegap ist. Javascript ist asynchron. Das kann behilflich sein, wenn man Animationen erstellen will, aber wenn ein Befehl auf dem Resultat eines anderen Befehls aufbaut, kann es zu Problemen führen. Man kann das Problem mit Workarounds lösen. Zum einen besteht keine Garantie, dass diese Workarounds auf allen Geräten funktionieren, zum anderen ist es äusserst aufwendig einen zu schreiben. Also haben wir den Entschluss gefasst nochmals neu anzufangen, aber diesmal mit "`Android Studio"'. Android Studio nutzt Java als Programmiersprache. Auch mit dieser Programmiersprache waren wir geübt. Dies vereinfachte die Entwicklung der Funktionalität um einiges. Android Studio nutzt für die Benutzeroberfläche XML, welches syntaktisch identisch mit HTML ist. Also mussten wir nicht gross umdenken, um auch XML nutzen zu können. Bei XML gibt es andere Elemente, welche wir lernen mussten. Etwas gewöhnungsbedürftig war die Verknüpfung zwischen XML und Java. Wir mussten lernen, wie man mit Java auf Elemente vom XML zugreift und diese verändert. Hinzu kam, dass wir uns mit dem Lebenszyklus einer App auseinandersetzen mussten. Bei Phonegap wurde das automatisch erledigt. Nach dem Lernen dieser Kleinigkeiten ging die Entwicklung viel schneller voran als bei Phonegap. Somit hat sich der Umstieg definitiv gelohnt.
