Es gibt mehrere Möglichkeiten eine Android App zu programmieren. Man kann ein Framework (z.B. "`LibGDX"') nutzen um mit einer vertrauten Programmiersprache und IDE eine App zu entwickeln. Google selbst bietet eine eigene IDE namens "`Android Studio"'. Wir entschieden uns anfangs mit Phonegap zu entwickeln.


\subsection{Die Arbeit mit Phonegap}
Phonegap ist mit dem Framework Cordova aufgebaut. Der Vorteil von Phonegap ist, dass es für das Design HTML nutzt. Wir sind vertraut mit HTML und können damit schnell relativ ansehnliche Benutzeroberflächen (kurz: GUI) erstellen. Als Programmiersprache nutzt Phonegap Javascript. Auch damit sind wir vertraut, da wir ein Semester lang damit gearbeitet haben. Wir konnten also direkt loslegen. Phonegap bietet zudem die Möglichkeit die App direkt auf dem Smartphone zu betrachten. Diese Funktion ist sehr praktisch, wenn man kurz was betrachten will. 

\subsection{Der Aufbau der App}
Wir wollten die Arbeit am Aufbau der App gleichmässig aufteilen. Wenn mehrere Personen an einem Projekt arbeiten, muss man dafür sorgen, dass keine Konflikte entstehen. Konflikte können entstehen, wenn z.B. mehrere Personen die gleiche Datei bearbeiten. Um dies zu verhindern wollten wir jede Seite der App in eine eigene Datei auslagern. Es sollte ein Menu geben, von wo aus man auf jede Seite zugreifen konnte. Dieses Menu musste am Anfang von einer Person erstellt werden. Danach konnten wir die einzelnen Seiten auf die Personen aufteilen und gleichzeitig Arbeiten, ohne dass Konflikte entstehen. Jede Seite sollte eine eigene Funktion haben. Z.B. gab es eine Seite für die Abfragen, eine Seite mit Favoriten etc. Mit einem Zurück-Knopf kommt man von jeder Seite zum Menu zurück.

\subsection{Der Wechsel zu "`Android Studio"'}
Wir haben viel Zeit damit verbracht mit Phonegap eine schöne Benutzeroberfläche zu erstellen. Als wir dann dazu kamen die Funktionalität zu programmieren, merkten wir was der Nachteil von Phonegap ist. Phonegap nutzt Javascript als Programmiersprache. Javascript ist asynchron. Das heisst, dass es nicht linear (Ein Befehl nach dem Anderen) ausgeführt wird, sondern die Befehle einer Funktion praktisch gleichzeitig ausführt. Dies ist super, wenn man Animationen erstellen will, aber wenn ein Befehl auf das Resultat von einem anderen Befehl aufbaut, gibt es viele Probleme. Man kann das Problem mit Workarounds lösen, aber dies ist einerseits möglicherweise stabil auf allen Geräten und andererseits ist es extrem aufwändig. Also fassten wir den Entschluss nochmals neu anzufangen, aber diesmal mit "`Android Studio"'. "`Android Studio"' nutzt Java als Programmiersprache. Auch mit dieser Programmiersprache kennen wir uns bestens aus. Dies vereinfachte die Entwicklung der Funktionalität um einiges. "`Android Studio"' nutzt für die GUI XML. XML ist von der Syntax her praktisch gleich wie HTML. Also mussten wir nicht viel umlernen. Bei XML gab es nur andere Elemente, welche wir lernen mussten. Etwas gewöhnungsbedürftig war die Verknüpfung zwischen XML und Java. Wir mussten lernen, wie man mit Java auf Elemente vom XML zugreift und diese verändert. Hinzu kam noch, dass wir uns mit dem Lebenszyklus einer App auseinandersetzen mussten. Bei Phonegap wurde das automatisch erledigt. Nach dem Lernen dieser Kleinigkeiten ging die Entwicklung viel schneller voran als bei Phonegap. Somit hat sich der Umstieg definitiv gelohnt. 