Das Thema unserer Projektarbeit musste im sich im Bereich der "Daten und Informationen" befinden. Wir entschieden uns etwas zu programmieren, weil das eine unserer Leidenschaften ist. Durch verschiedene Projekte konnten wir schon eine gewisse Menge an Erfahrung sammeln. Wir hatten die Idee eine App zu entwickeln um uns auf Neuland wagen zu können und damit wir unseren eigenen Horizont erweiteren konnten. Da 2 von 3 ein Ger\"at mit dem Betriebssystem Android besitzt und wir schon eine Android-Entwickler-Lizenz besassen, entschieden wir uns die App auf der basis von Android zu schreiben. Die App sollte sowohl funktional sein als auch von uns selbst verwendet werden. Wir sind zudem ambitionierte "Magic: The Gathering"-Spieler. Bei "Magic: The Gathering" (kurz: MTG) werden mehrmals im Jahr neue Kartensets ver\"offentlicht. Die darin enthaltenen Karten haben neue Namen, Bilder, Fähigkeiten und Attribute. Man muss diese aber erst kennenlernen, bevor man mit ihnen richtig spielen kann. Bisher war es relativ m\"uhsam die Karten auswendig zu lernen, wir keine Software zum Lernen der Karten gefunden haben. Dies ist allerdings n\"otig, um ein schnelles und fl\"ussiges Spielen zu ermöglichen. Wir liessen uns von Quizlet und Memorize inspirieren, welche Fremdw\"orter abfragen. Dadurch kamen wir auf die Idee eine App zu entwickeln, welche die aktuellen Karten dem Benutzer beibringt. Demnach kann man viel Zeit einsparen in der man die Karten nicht suchen muss. Folglich haben wir ein technisch anspruchvolles Projekt, welches wir auch privat nutzen k\"onnen.