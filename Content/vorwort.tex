Das Thema unserer Projektarbeit musste sich im Bereich von "Daten und Informationen" liegen. Dazu wollten wir etwas programmieren, da wir interessierte Programmierer sind und das Programmieren uns grosse Freude bereitet. Durch verschiedene Projekte hatten wir schon eine gewisse Menge an Erfahrung gesammelt. Wir kamen auf die Idee, eine App zu entwickeln, da wir noch nie eine App programmiert haben und wir somit neues lernen konnten. Da 2 von 3 ein Ger\"at mit dem Betriebssystem Android besitzt und wir schon eine Android-Entwickler-Lizenz hatten, entschieden wir uns die App f\"ur Android zu entwickeln. Die App sollte funktional sein und wir sollten Verwendung f\"ur die App haben. Wir sind zudem ambitionierte "Magic: The Gathering"-Spieler. Bei "Magic: The Gathering" (kurz: MTG) werden mehrmals im Jahr ein neues Set Karten ver\"offentlicht. Diese haben neue Namen, Bilder, Fähigkeiten und Attribute. Somit muss man diese erst kennenlernen, bevor man mit ihnen richtig spielen kann. Bisher war es relativ m\"uhsam die Karten auswendig zu lernen, wir keine Lernwebsites oder Apps gefunden haben, die einem das Lernen erleichtern. Dies ist allerdings n\"otig, um ein schnelles und fl\"ussiges Spielen zu sichern. Wir liessen uns von Quizlet inspirieren, welches Fremdw\"orter abfragt. Dadurch kamen wir auf die Idee eine App zu entwickeln, welches die aktuellen Karten dem Benutzer beibringt. Demnach spart man Zeit, da man die Karten nicht suchen muss. Folglich haben wir ein technisch anspruchvolles Projekt, welches wir auch privat nutzen k\"onnen.