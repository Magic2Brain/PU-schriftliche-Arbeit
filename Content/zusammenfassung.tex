Unsere Projektidee entstand mit dem Hinblick auf neue und bestehende Turnierspieler der Kartenspiels Magic: The Gathering (\textit{kurz: MTG}).
Ungefähr alle 2-3 Monate kommt ein neues Kartenpaket heraus. Dieses wird Set genannt. Die Karten in jedem Set haben andere Eigenschaften in der Angriffsstärke, der Verteidigung oder haben sogar besondere oder einzigartige Eigenschaften. Um nicht ständig diese Daten von den Karten zu lesen wollten wir eine App machen, die genau diese Daten deren Nutzern beibringt.
Ein Spiel hat nämlich auch eine zeitliche Begrenzung, die in einem Unentschieden endet. Daher ist es für beide Spieler von Vorteil wenn sie schnell spielen.

Um dies zu bewerkstelligen haben wir zahlreiche APIs \textit{(\textbf{A}pplication \textbf{P}programming \textbf{I}nterface)} verglichen. Zu beachten gab es 3 grundlegende Komponenten: Bilder der Karte, Text mit Eigenschafen der Karte und Eine Möglichkeit den Fortschritt zu speichern. Die Bilder haben wir direkt von dem Kartenbrowser von Wizards of the Coast genommen. Als API für die Karteninformationen hat MTGJSON gesiegt, wobei dies keine API ist, sondern eine Downloadquelle für Karten im JSON Format. JSON ist eine schnelle und einfach zu begreifende Alternative zu XML. Somit kann die App auch offline verwendet werden. Bilder, welche einmal heruntergeladen wurden, werden bis zum Deinstallieren der App oder bis zum manuellen Löschen auf dem Gerät des Benutzers gespeichert. Informationen über den Fortschritt werden in den \textit{SharedPreferences} von Android gespeichert. Dies kann man sich wie ein Karteisystem vorstellen. Jede App erhält ihr eigenes Fach und kann dort Informationen ablegen.

Ursprünglich wollten wir unsere App in HTML, CSS und JavaScript schreiben, welche wir dann mit Cordova zu einer Android und IOS App kompilliert hätten.
Jedoch wurde uns durch die Natur von JavaScipt verwehrt einige essentielle Funktionen zu implementieren. Dies lag vorwiegend daran, dass JavaScript asynchron arbeitet und man dies nicht mehr wie damals ausschalten kann. Asynchron beschreibt die Arbeitsweise von einer Programmiersprache. In diesem Fall werden mehrere Funktionen gleichzeitig ausgeführt, was ein gut durchdachtes und stabiles Programm erfordert. Ausserdem müssen in einigen Fällen \textit{"Promises"} zurückgegeben werden. Das sind Verweise auf einen Wert, der in der Zukunft existieren wird, aber noch nicht bekannt ist wann es so weit ist. Als Beispiel nehme man eine ladende Webseite. Niemand weiss wann sie fertig geladen ist, da das von der Verbindungsgeschwindikeit abhängt, aber der Fakt, dass sie einmal geladen sein wird ist unbestritten. Promises kann man auch nicht ohne weiters abwarten bzw. \textit{auflösen}, weil das in sehr vielen Fällen kritische Auswirkungen auf die Geschwindigkeit des Programms hat. Synchrone Sprachen hingegen arbeiten den Programmcode wie eine Art Stapel ab. Zeile für Zeile liest der Compiler den Code durch und wandelt ihn in Maschienensprache um. Das lässt dem Programmierer wenig Spielraum für Fehler übrig. Weil die Zeit schon relativ fortgeschritten und die App gröstenteil fertig war entschieden wir uns nur schweren Herzens dazu unser Projekt aufzugeben und die App in Java zu schreiben. In Java schritten wir jedoch viel schneller voran als in JavaScript.